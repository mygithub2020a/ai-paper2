% Introduction

Optimization of neural networks remains a central challenge in machine learning, with
algorithm choice significantly impacting training efficiency and final performance.
While first-order methods like SGD and Adam dominate practice, there is ongoing
interest in novel optimization principles that may offer advantages for specific
problem classes or provide theoretical insights.

In this work, we explore whether principles from \emph{quantum filtering theory}---
specifically the Belavkin equation for optimal quantum state estimation---can inspire
new neural network optimizers. The Belavkin equation describes how to optimally update
a quantum state given continuous noisy measurements, balancing information gain against
measurement-induced disturbance. This naturally suggests an optimization framework where:
\begin{itemize}
\item Parameters are analogous to quantum states
\item Gradients are analogous to measurement signals
\item Learning involves balancing exploration (uncertainty) with exploitation (gradient descent)
\end{itemize}

\paragraph{Key Challenges}
Direct application of quantum filtering to classical neural networks faces fundamental obstacles:
(1) density matrices scale as $O(d^2)$ for $d$ parameters---prohibitive for modern networks,
(2) the classical-quantum analogy is heuristic rather than rigorous, and
(3) most neural network learning does not naturally fit the continuous measurement framework.

\paragraph{Our Approach}
We develop a \emph{tractable approximation} that captures core principles while remaining
computationally practical. Our optimizer implements the update rule:
\begin{equation}
\params_{t+1} = \params_t - [\damping \cdot (\grad \loss)^2 + \eta \cdot \grad \loss]\,dt
                + \exploration \cdot \grad \loss \cdot \sqrt{dt} \cdot \epsilon_t
\end{equation}
where $\damping$ controls gradient-dependent damping, $\eta$ is the learning rate, and
$\exploration$ controls multiplicative noise scaled by gradient magnitude.

\paragraph{Contributions}
\begin{enumerate}
\item A novel optimizer derived from quantum filtering principles with efficient
      $O(n)$ per-step complexity
\item Adaptive variants that automatically tune hyperparameters based on gradient statistics
\item Comprehensive benchmarks on tasks exhibiting phase transitions and grokking
\item Ablation studies isolating contributions of damping, noise, and adaptation
\item Honest assessment of when quantum inspiration helps vs. standard methods
\end{enumerate}

\TODO{Update with actual experimental findings once results are available}
