% Discussion

\subsection{When Does Belavkin Help?}

Based on our experiments, the Belavkin optimizer shows advantages in [SCENARIOS TBD]:

\TODO{Fill in based on actual results. Be honest about limitations.}

Potential explanations:
\begin{itemize}
\item The gradient-dependent damping provides adaptive regularization
\item Multiplicative noise helps escape poor local minima in structured problems
\item The quantum-inspired balance of exploitation/exploration matches certain loss landscapes
\end{itemize}

\subsection{Limitations}

We emphasize several important limitations:

\paragraph{Classical Heuristic}
Our optimizer is a \emph{heuristic inspired by} quantum filtering, not an application
of quantum computing. The analogy between neural network optimization and quantum
state estimation is suggestive but not rigorous.

\paragraph{Additional Hyperparameters}
Belavkin introduces two new hyperparameters ($\damping$, $\exploration$) beyond
learning rate. While adaptive variants reduce tuning burden, this adds complexity.

\paragraph{Task Specificity}
Performance appears task-dependent. The optimizer may excel on structured problems
with phase transitions but offer less advantage on standard supervised learning.

\paragraph{Scalability}
We have not yet tested on very large models (GPT-scale). The $O(p)$ complexity is
promising, but practical benefits at scale remain to be demonstrated.

\subsection{Theoretical Open Questions}

Several theoretical questions merit further investigation:

\begin{enumerate}
\item \textbf{Convergence guarantees}: Under what assumptions does Belavkin converge?
      Can we derive rates (e.g., $O(1/t)$ or $O(1/\sqrt{t})$)?

\item \textbf{Connection to existing theory}: What is the precise relationship to:
      \begin{itemize}
      \item Natural gradient descent via Fisher information?
      \item Langevin dynamics and sampling-based optimization?
      \item Information geometry?
      \end{itemize}

\item \textbf{Generalization bounds}: Does the quantum-inspired noise provide
      implicit regularization? Can we prove PAC-Bayes style generalization bounds?

\item \textbf{Optimal hyperparameters}: Is there a principled way to set $\damping$
      and $\exploration$ based on problem structure?
\end{enumerate}

\subsection{Broader Impact}

This work explores novel optimization principles inspired by quantum physics applied
to classical machine learning. Potential societal impacts are similar to other
optimization research: improved training efficiency could reduce computational costs
and environmental impact, but could also enable scaling to larger models with
associated risks.

We have made all code and data publicly available to support reproducibility and
further research.
