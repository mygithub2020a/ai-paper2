\documentclass{article}
\usepackage{graphicx}
\usepackage{amsmath}
\usepackage{geometry}
\geometry{a4paper, margin=1in}

\title{The Belavkin Optimizer: A Novel Approach to Optimization}
\author{Jules}
\date{\today}

\begin{document}

\maketitle

\begin{abstract}
This paper introduces the Belavkin Optimizer, a novel optimization algorithm derived from the Belavkin quantum filtering equation. We present the core update rule of the optimizer and benchmark its performance against standard optimizers like Adam, SGD, and RMSprop on synthetic datasets. Our results show that the Belavkin Optimizer is a promising new approach to optimization, with potential applications in deep learning and reinforcement learning.
\end{abstract}

\section{Introduction}
The field of optimization is central to machine learning and deep learning. The choice of optimizer can have a significant impact on the training time and performance of a model. In this paper, we propose a new optimizer, the Belavkin Optimizer, which is based on the Belavkin equation from quantum filtering theory.

\section{Methods}
The Belavkin Optimizer is based on the following update rule:
\begin{equation}
d\theta = -[\gamma \cdot (\nabla L(\theta))^2 + \eta \cdot \nabla L(\theta)] + \beta \cdot \nabla L(\theta) \cdot \epsilon
\end{equation}
where $\gamma$ is the adaptive damping factor, $\eta$ is the learning rate, $\beta$ is the stochastic exploration factor, and $\epsilon$ is a random variable.

\subsection{Implementation}
We implemented the Belavkin Optimizer in PyTorch. The implementation is straightforward and follows the update rule described above.

\section{Results}
We benchmarked the Belavkin Optimizer against Adam, SGD, and RMSprop on two synthetic datasets: modular arithmetic and modular composition. The results are shown in Figure~\ref{fig:benchmark}.

\begin{figure}[h!]
\centering
\includegraphics[width=\textwidth]{benchmark_results.png}
\caption{Benchmark results on synthetic datasets.}
\label{fig:benchmark}
\end{figure}

\section{Conclusion}
The Belavkin Optimizer is a new and promising optimization algorithm. Our initial results show that it is competitive with existing optimizers on synthetic datasets. Further research is needed to explore its performance on real-world datasets and to investigate its theoretical properties.

\end{document}
